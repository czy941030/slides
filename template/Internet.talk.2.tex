\documentclass[mathserif]{beamer}

% 设置字体
\usepackage{xeCJK}
\usepackage{fontspec}
\usepackage{xunicode}
\usepackage{xltxtra}
\setCJKmainfont{Hiragino Sans GB W3}
\xeCJKsetup{
  CJKspace = true,
  CJKmath = true,
  CheckSingle = true,
  AutoFallBack = true,
  AutoFakeBold = true,
  AutoFakeSlant = true,
  PunctStyle = kaiming,
  AllowBreakBetweenPuncts = true
}

% 图片支持
\usepackage{graphicx}

% 主题
\usetheme[hideothersubsections,left]{PaloAlto}
\usecolortheme{seahorse}
\setbeamercovered{transparent}	%设置遮盖透明
%\logo{\includegraphics[height=1.4cm]{./logo.png}}


\begin{document}

\title{命名数据网络\thanks{powered by \XeLaTeX}\\Named Data Networking}
\author{徐磊\thanks{\href{mailto:leopard.lie@gmail.com}{leopard.lie@gmail.com}}、张宇翔\thanks{\href{mailto:zz593141477@gmail.com}{zz593141477@gmail.com}}}
\institute{清华大学·计算机科学与技术系·计33}
\date{2014年4月}

\begin{frame}
  \titlepage
\end{frame}

\AtBeginSection{
  \begin{frame}
    \frametitle{摘要}
    \tableofcontents[currentsection,hideothersubsections]
  \end{frame}
}

\section{简介}
\subsection{设计原则}
\begin{frame}{设计原则}
  基于对目前互联网优缺点的分析:
  \begin{itemize}
  \item 保留沙漏模型
  \item 完善网络安全性
  \item 保留和拓展端到端的原则
  \item 网络流量的自控制
  \item 保持路由策略与路由分离
  \end{itemize}
\end{frame}

\subsection{一个例子}
\begin{frame}{一个例子}
  \begin{itemize}
  \item<1> 用户想要获取数据/thu/doc/net.pdf
  \item<2> 发送名为/thu/doc/net.pdf的兴趣包
  \item<3> 路由器记下兴趣包来源的接口
  \item<4> 在路由表中寻找最长前缀进行转发
  \end{itemize}
\end{frame}

\begin{frame}{一个例子}
  \begin{itemize}
  \item<1> 兴趣包到达有该数据的终端
  \item<2> 名为/thu/doc/net.pdf的数据包被返回
  \item<3> 根据路由器之前的记录来返回数据包
  \end{itemize}
\end{frame}

\begin{frame}{一个例子}
  \begin{itemize}
  \item 兴趣包和数据包都不包含主机和接口信息
  \item 兴趣包根据它的名称来寻找主机
  \item 路由器可以缓存兴趣包和数据包来提高效率
  \end{itemize}
\end{frame}

\section{体系结构}
\subsection{数据包}
\begin{frame}{数据包}
  \uncover<1-6>{}
  \begin{columns}
    \begin{column}[t]{0.5\textwidth}
      兴趣包 Interest Pack
      \begin{itemize}
      \item 内容名 Content Name
      \item 选择选项 Selector
        \only<2>{
          \begin{itemize}
          \item 优选顺序\\ order preference
          \item 发布者过滤\\ publisher filter
          \item 选择范围\\ scope
          \end{itemize}
        }
      \item 随机时间值 Nonce
      \end{itemize}
    \end{column}

    \begin{column}[t]{0.5\textwidth}
      数据包 Data Pack
      \begin{itemize}
      \item 内容名 Content Name
      \item 签名 Signature
        \only<4>{
          \begin{itemize}
          \item 摘要算法\\ digest algorithm
          \item 证明\\ witness
          \end{itemize}
        }
      \item 签署信息 Signed Info
        \only<5>{
          \begin{itemize}
          \item 发布ID\\ publisher ID
          \item 密钥定位\\ key locator
          \item 失效时间\\ stale time
          \end{itemize}
        }
      \item 数据 Data
      \end{itemize}
    \end{column}
  \end{columns}
\end{frame}

\subsection{命名}
\begin{frame}{命名}
  \begin{itemize}
  \item 命名对于网络不透明(路由器不关心名称的含义)
  \item 使用分层的结构化命名
  \item 接收者必须能够构造所请求数据的命名
  \item 双方约定命名算法、由部分名称构造(/parc/videos/abc.mpg -> /parc/videos/abc.mpg/1/1)
  \end{itemize}
\end{frame}

\subsection{安全}
\begin{frame}{安全}
  \begin{itemize}
  \item 构建在数据上的安全机制
  \item 每条数据和名称都带有数字签名
  \item 可以进一步扩展访问权限控制机制(签名中可以包含身份信息)
  \item 挑战
    \begin{itemize}
    \item 类DoS的兴趣包洪水攻击
    \item 内容污染攻击
    \item 缓存、命名、签名的隐私保护
    \end{itemize}
  \end{itemize}
\end{frame}

\subsection{路由}
\begin{frame}{路由}
  \begin{itemize}
    \item 路由表中匹配最长前缀转发
    \item PIT缓存兴趣包
    \item 数据包由原路径反向回传
    \item 允许把一个请求向多个接口转发
  \end{itemize}
\end{frame}

\subsection{缓存}
\begin{frame}{缓存}
  \begin{itemize}
    \item 根据数据的名称,缓存曾经返回的数据包
    \item 缓存中不包含请求者信息(现有的IP网络中,头会泄露信息)
    \item 缓存策略(LRU、LFU)
    \item 防止缓存污染
  \end{itemize}
\end{frame}

\section{DEMO}
\begin{frame}
%\includegraphics[width = \textwidth]{pic.png}
\end{frame}
\AtBeginSection{}

\section[End]{}
\begin{frame}{End}
  \color{blue}\huge{Thanks For Attention!}
\end{frame}

\end{document}
