\documentclass[mathserif]{beamer}

\mode<presentation> {
	\usetheme{PaloAlto}
	\usecolortheme{whale}
}


\usepackage{xeCJK}
\setCJKmainfont{WenQuanYi Micro Hei}

\usepackage{graphicx}
\usepackage{booktabs}
\usepackage{fontspec}
\usepackage{xunicode}
\usepackage{xltxtra}

\usepackage{amsmath,amssymb}

\setbeamercovered{transparent}
\logo{\includegraphics[height=1.58cm]{./logo.jpg}}

%----------------------------------------------------------
%	Title Page
%----------------------------------------------------------

\title[Finding TPMFP in BTD]{Finding Time Period-Based Most Frequent Path in Big Trajectory Data\thanks{powered by \XeLaTeX} }

\author{Ziyang Chen}
\institute[FDU]
{
	Fudan University\\
	\medskip
	\textit{13307130148@fudan.edu.cn}
}
\date{\today}

\begin{document}
\newtheorem{property}[theorem]{\textsc{Property}}

\begin{frame}
\titlepage
\end{frame}

\AtBeginSection{
	\begin{frame}
	\frametitle{Summary}
	\tableofcontents
	\end{frame}
}

\section{Overview}
\begin{frame}{Overview}
	\begin{itemize}
	\item The main task: find \textit{the most frequent}(MFP) during user-specified time periods in large-scale historical trajectory data.
	\item They refer to this query as \textit{time period-based MFP}(TPMFP).
	\item Specifically, given a time peroid $T$, a source $v_s$ and a destination $v_d$, TPMFP searchs the MFP from $v_s$ to $v_d$ during $T$.
	\end{itemize}
\end{frame}

\begin{frame}{Overview}
	\begin{itemize}
	\item None of the previous work can well reflect people's common sense notion which can be described by the following key properties:\\
		\begin{itemize}
		\item \textit{suffix-optimal}
		\item \textit{length-insensitive}
		\item \textit{bottleneck-free}
		\end{itemize}
	\item The first task is to give a TPMFP definition that satisfies the above three properties.
	\item The next task is to find TPMFP over huge amount of trajectory data efficiently.(over $11,000,000$ trajectories.)
	\end{itemize}
\end{frame}

\section{Key Properities}
\begin{frame}{Key Properities}
	\begin{property}[\textsc{Suffix-Optimal}]
	Let $P^*$ denote the $v_S-v_d$ MFP. For any vertex $u \in P^*$ , the sub-path (suffix) of $P^*$ from $u$ to $v_d$ should be the $u–v_d$ MFP.
	\end{property}
	\begin{property}[\textsc{Length-Insensitive}]
	The length of any path should not be a deciding factor of whether it is the $v_s-v_d$ MPF.
	\end{property}
	\begin{property}[\textsc{Bottleneck-Free}]
	The MPF $P^*$ should not contain infrequent edges(i.e., bottlenecks).
	\end{property}

\end{frame}

\end{document}
